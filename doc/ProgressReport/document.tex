\documentclass[a4paper,twoside]{article}
\usepackage[T1]{fontenc}
\usepackage[bahasa]{babel}
\usepackage{graphicx}
\usepackage{graphics}
\usepackage{float}
\usepackage[cm]{fullpage}
\pagestyle{myheadings}
\usepackage{etoolbox}
\usepackage{setspace} 
\usepackage{lipsum} 
\setlength{\headsep}{30pt}
\usepackage[inner=2cm,outer=2.5cm,top=2.5cm,bottom=2cm]{geometry} %margin
% \pagestyle{empty}

\makeatletter
\renewcommand{\@maketitle} {\begin{center} {\LARGE \textbf{ \textsc{\@title}} \par} \bigskip {\large \textbf{\textsc{\@author}} }\end{center} }
\renewcommand{\thispagestyle}[1]{}
\markright{\textbf{\textsc{Laporan Perkembangan Pengerjaan Skripsi\textemdash Sem. Genap 2015/2016}}}

\onehalfspacing
 
\begin{document}

\title{\@judultopik}
\author{\nama \textendash \@npm} 

%ISILAH DATA BERIKUT INI:
\newcommand{\nama}{Ellena Angelica}
\newcommand{\@npm}{2015730029}
\newcommand{\tanggal}{15/11/2018} %Tanggal pembuatan dokumen
\newcommand{\@judultopik}{Kolektor Pengumuman Informatika} % Judul/topik anda
\newcommand{\kodetopik}{}
\newcommand{\jumpemb}{1} % Jumlah pembimbing, 1 atau 2
\newcommand{\pembA}{Pascal Alfadian Nugroho}
\newcommand{\pembB}{-}
\newcommand{\semesterPertama}{40 - Ganjil 18/19} % semester pertama kali topik diambil, angka 1 dimulai dari sem Ganjil 96/97
\newcommand{\lamaSkripsi}{1} % Jumlah semester untuk mengerjakan skripsi s.d. dokumen ini dibuat
\newcommand{\kulPertama}{Skripsi 1} % Kuliah dimana topik ini diambil pertama kali
\newcommand{\tipePR}{B} % tipe progress report :
% A : dokumen pendukung untuk pengambilan ke-2 di Skripsi 1
% B : dokumen untuk reviewer pada presentasi dan review Skripsi 1
% C : dokumen pendukung untuk pengambilan ke-2 di Skripsi 2

% Dokumen hasil template ini harus dicetak bolak-balik !!!!

\maketitle

\pagenumbering{arabic}

\section{Data Skripsi} %TIDAK PERLU MENGUBAH BAGIAN INI !!!
Pembimbing utama/tunggal: {\bf \pembA}\\
Pembimbing pendamping: {\bf \pembB}\\
Kode Topik : {\bf \kodetopik}\\
Topik ini sudah dikerjakan selama : {\bf \lamaSkripsi} semester\\
Pengambilan pertama kali topik ini pada : Semester {\bf \semesterPertama} \\
Pengambilan pertama kali topik ini di kuliah : {\bf \kulPertama} \\
Tipe Laporan : {\bf \tipePR} -
\ifdefstring{\tipePR}{A}{
			Dokumen pendukung untuk {\BF pengambilan ke-2 di Skripsi 1} }
		{
		\ifdefstring{\tipePR}{B} {
				Dokumen untuk reviewer pada presentasi dan {\bf review Skripsi 1}}
			{	Dokumen pendukung untuk {\bf pengambilan ke-2 di Skripsi 2}}
		}
		
\section{Latar Belakang}
Pengumuman di jurusan Teknik Informatika UNPAR pada umumnya dilakukan oleh email. Pengumuman lewat email ini praktis karena tidak perlu menunggu sampai email sampai ke tujuan dan dijamin sampai ke tujuan. Selain itu, konten yang disampaikan melalui email fleksibel. Konten tidak harus hanya tulisan tapi dapat menambahkan lampiran, mengubah gaya tulisan dan lain-lain.

Namun, email kurang terorganisir dengan baik. Email yang masuk sering tercampur dengan email lain sehingga mahasiswa kesulitan mencari email yang penting. Dampaknya, pengumuman-pengumuman penting sering tidak terbaca secara tidak sengaja.

Pada skripsi ini, akan dibuat solusi masalah tadi dengan membuat suatu fitur. Fitur ini akan menangkap email-email pengumuman yang masuk ke sebuah email khusus untuk menangkap pengumuman. Pertama email yang masuk ke email khusus akan diperiksa pengirimnya. Apabila pengirim adalah email yang terdaftar sebagai email yang berhak melakukan pengumuman maka email tersebut adalah email pengumuman. Setelah itu email tersebut akan dibuatkan permanent link dan disisipkan pada basis data. Lalu, mahasiswa akan menerima permanent link tersebut melalui notifikasi dari akun Line@. Line@ adalah layanan dari Line Corporation yang memudahkan pemilik bisnis atau organisasi menyampaikan pesan kepada pengikutnya di aplikasi pengirim pesan Line.

Fitur ini akan dibangun sebagai fitur tambahan pada BlueTape, sebuah website milik jurusan teknik Informatika Unpar. Pembangunan fitur ini membutuhkan modifikasi BlueTape sehingga dapat dijalankan di Heroku dan menggunakan basis data PostgreSQL. Heroku adalah layanan yang memungkinkan pengembang membangun, menjalankan, dan mengoperasikan aplikasi di dalam internet. Selain itu, sistem ini membutuhkan beberapa fitur dari PHP IMAP dan layanan pengirim pesan instan Line@.

\section{Tujuan}
\begin{itemize}
\item Melakukan perawatan pada BlueTape agar dapat mendukung fitur kolektor pengumuman dengan bantuan Heroku dan PostgreSQL
\item Mengimplementasikan fitur kolektor pengumuman pada BlueTape
\end{itemize}

\section{Rumusan Masalah}
\begin{itemize}
\item Bagaimana cara mengubah BlueTape agar dapat mendukung fitur kolektor pengumuman dengan bantuan Heroku dan PostgreSQL ?
\item Bagaimana cara mengimplementasikan kolektor pengumuman pada BlueTape ?
\end{itemize}

\section{Detail Perkembangan Pengerjaan Skripsi}
Detail bagian pekerjaan skripsi sesuai dengan rencan kerja/laporan perkembangan terkahir :
	\begin{enumerate}
		\item \textbf{Melakukan studi literatur tentang GMail, Line, Heroku dan PostgreSQL}\\
		{\bf Status :} Ada sejak rencana kerja skripsi.\\
		{\bf Hasil :} S
		\item \textbf{Memodifikasi BlueTape sehingga dapat menangkap email yang masuk ke email khusus}\\
		{\bf Status :} Ada sejak rencana kerja skripsi.\\
		{\bf Hasil :} S
		\item \textbf{Memodifikasi BlueTape sehingga dapat melakukan push notification ke akun Line@}\\
		{\bf Status :} Ada sejak rencana kerja skripsi.\\
		{\bf Hasil :} S
		\item \textbf{Memodifikasi BlueTape sehingga dapat berjalan di Heroku menggunakan PostgreSQL}\\
		{\bf Status :} Ada sejak rencana kerja skripsi.\\
		{\bf Hasil :} S
		\item \textbf{Melakukan pengujian}\\
		{\bf Status :} Ada sejak rencana kerja skripsi.\\
		{\bf Hasil :} S
		\item \textbf{Menulis dokumen skripsi}\\
		{\bf Status :} Ada sejak rencana kerja skripsi.\\
		{\bf Hasil :} S
	\end{enumerate}

\section{Pencapaian Rencana Kerja}
Langkah-langkah kerja yang berhasil diselesaikan dalam Skripsi 1 ini adalah sebagai berikut:
\begin{enumerate}
\item
\item
\item
\end{enumerate}



\section{Kendala yang Dihadapi}
%TULISKAN BAGIAN INI JIKA DOKUMEN ANDA TIPE A ATAU C
Kendala - kendala yang dihadapi selama mengerjakan skripsi :
\begin{itemize}
	\item Terlalu banyak melakukan prokratinasi
	\item Terlalu banyak godaan berupa hiburan (game, film, dll)
	\item Skripsi diambil bersamaan dengan kuliah ASD karena selama 5 semester pertama kuliah tersebut sangat dihindari dan tidak diambil, dan selama 4 semester terakhir kuliah tersebut selalu mendapat nilai E
	\item Mengalami kesulitan pada saat sudah mulai membuat program komputer karena selama ini selalu dibantu teman
\end{itemize}

\vspace{1cm}
\centering Bandung, \tanggal\\
\vspace{2cm} \nama \\ 
\vspace{1cm}

Menyetujui, \\
\ifdefstring{\jumpemb}{2}{
\vspace{1.5cm}
\begin{centering} Menyetujui,\\ \end{centering} \vspace{0.75cm}
\begin{minipage}[b]{0.45\linewidth}
% \centering Bandung, \makebox[0.5cm]{\hrulefill}/\makebox[0.5cm]{\hrulefill}/2013 \\
\vspace{2cm} Nama: \pembA \\ Pembimbing Utama
\end{minipage} \hspace{0.5cm}
\begin{minipage}[b]{0.45\linewidth}
% \centering Bandung, \makebox[0.5cm]{\hrulefill}/\makebox[0.5cm]{\hrulefill}/2013\\
\vspace{2cm} Nama: \pemB \\ Pembimbing Pendamping
\end{minipage}
\vspace{0.5cm}
}{
% \centering Bandung, \makebox[0.5cm]{\hrulefill}/\makebox[0.5cm]{\hrulefill}/2013\\
\vspace{2cm} Nama: \pembA \\ Pembimbing Tunggal
}
\end{document}


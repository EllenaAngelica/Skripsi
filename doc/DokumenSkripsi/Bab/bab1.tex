%versi 2 (8-10-2016) 
\chapter{Pendahuluan}
\label{chap:pendahuluan}
   
\section{Latar Belakang}
\label{sec:latarBelakang}
Pengumuman di jurusan Teknik Informatika UNPAR pada umumnya dilakukan lewat \textit{email}. Pengumuman lewat \textit{email} lebih praktis daripada pengumuman di papan pengumuman karena \textit{email} dijamin sampai ke pihak yang dituju setelah dikirim. Namun, layanan \textit{email} memiliki kotak masuk yang kurang terorganisir. Berbagai macam \textit{email} yang masuk tercampur di kotak masuk sehingga dapat menyulitkan pemilik \textit{email} untuk mencari \textit{email} yang penting. Hal ini juga dapat mengakibatkan pengumuman-pengumuman penting tidak dibaca secara tidak sengaja.

Penelitian pada skripsi ini hendak membuat solusi untuk kekurangan \textit{email} tersebut dengan cara membuat fitur kolektor pengumuman di BlueTape. Fitur ini akan mengumpulkan \textit{email} yang berisi pengumuman dan menampilkannya di BlueTape. Agar penerima pengumuman tahu ada pengumuman baru di BlueTape, fitur ini juga akan memanfaatkan layanan LINE@. LINE@ adalah layanan dari \textit{Line Corporation} yang memungkinkan pemilik bisnis atau organisasi membuat akun khusus (disebut akun LINE@) yang dapat mengirim pesan ke banyak pengikut secara bersamaan. Penerima pengumuman akan diminta untuk mengikuti akun LINE@ BlueTape sehingga dapat menerima notifikasi LINE apabila ada pengumuman baru. Agar BlueTape versi skripsi ini dapat diakses melalui internet, fitur ini juga akan memanfaatkan layanan Heroku.

\section{Rumusan Masalah}
\label{sec:rumusanmasalah}
Rumusan masalah dari skripsi ini adalah: 
\begin{itemize}
	\item Bagaimana cara memodifikasi BlueTape agar dapat berjalan di Heroku?
	\item Bagaimana cara mengimplementasikan fitur kolektor pengumuman pada BlueTape?
\end{itemize}

\section{Tujuan}
\label{sec:tujuan}
Tujuan dari skripsi ini adalah:
\begin{itemize}
	\item Memodifikasi BlueTape agar dapat berjalan di Heroku.
	\item Mengimplementasikan fitur kolektor pengumuman pada BlueTape.
\end{itemize}

\section{Batasan Masalah}
\label{sec:batasan}
Pada skripsi ini masalah dibatasi dengan batasan-batasan sebagai berikut:
\begin{itemize}
	\item Fitur ini tidak akan menampilkan lampiran karena dapat membuat masalah lebih kompleks.
	\item Pengumuman di BlueTape dapat dilihat oleh semua mahasiswa dan dosen yang memiliki hak akses ke BlueTape.
	\item Semua akun yang mengikuti akun LINE@ BlueTape dapat menerima notifikasi apabila ada pengumuman baru.
\end{itemize}

\section{Metodologi}
\label{sec:metodepenelitian}
Metode penelitian pada skripsi ini sebagai berikut:
	\begin{enumerate}
		\item Melakukan studi literatur tentang PHP IMAP, LINE@, dan Heroku.
		\item Memodifikasi BlueTape sehingga dapat mengumpulkan \textit{email} yang berisi pengumuman.
		\item Memodifikasi BlueTape sehingga dapat mengirim notifikasi ke akun Line@ BlueTape.
		\item Memodifikasi BlueTape sehingga dapat berjalan di Heroku.
		\item Melakukan pengujian.
		\item Menulis dokumen skripsi.
	\end{enumerate}

\section{Sistematika Pembahasan}
\label{sec:sispem}
Berikut adalah sistematika pembahasan skripsi ini :
\begin{enumerate}
\item Bab 1: Pendahuluan

Bab ini membahas gambaran umum dari skripsi.

\item Bab 2: Dasar Teori

Bab ini membahas dasar teori yang mendukung pembuatan skripsi ini.

\item Bab 3: Analisis

Bab ini membahas analisis yang dilakukan terhadap masalah yang diusung dalam skripsi ini.

\item Bab 4: Perancangan

Bab ini membahas perancangan sistem yang dibangun pada skripsi ini.

\item Bab 5: Implementasi dan Pengujian

Bab ini membahas hasil implementasi yang dilakukan beserta pengujian sistem.

\item Bab 6: Kesimpulan dan saran

Bab ini berisi kesimpulan yang didapat dari penelitian dan saran oleh penulis kepada pembaca yang hendak melanjutkan penelitian ini. 

\end{enumerate}
%versi 2 (8-10-2016) 
\chapter{Pendahuluan}
\label{chap:pendahuluan}
   
\section{Latar Belakang}
\label{sec:latarBelakang}
Pengumuman di jurusan Informatika pada umumnya dilakukan oleh email. Pengumuman lewat email ini praktis karena tidak perlu menunggu sampai email sampai ke tujuan dan dijamin sampai ke tujuan. Selain itu, konten yang disampaikan melalui email fleksibel. Konten tidak harus hanya tulisan tapi dapat menambahkan lampiran, mengubah gaya tulisan dan lain-lain.

Namun, mahasiswa zaman sekarang jarang membuka email. Selain itu, email yang masuk sering tercampur dengan email lain sehingga mahasiswa kesulitan mencari email yang penting. Dampaknya, pengumuman-pengumuman penting sering tidak tersampaikan kepada banyak mahasiswa.

Pada skripsi ini, akan dibuat solusi masalah tadi dengan membuat suatu sistem. Sistem ini akan menangkap email-email pengumuman. Email-email tersebut akan dibuatkan permanent link dan disisipkan pada basis data. Lalu, mahasiswa akan menerima permanent link tersebut melalui notifikasi dari akun Line@.

Sistem ini akan dibangun sebagai fitur tambahan pada BlueTape, sebuah website milik jurusan teknik Informatika Unpar. Pembangunan fitur ini membutuhkan modifikasi BlueTape sehingga dapat dijalankan di Heroku dan menggunakan basis data PostgreSQL. Selain itu, sistem ini membutuhkan beberapa fitur dari GMail dan Line.

\section{Rumusan Masalah}
\label{sec:rumusanmasalah}
\begin{itemize}
\item Apa saja yang harus dilakukan untuk menambah fitur kolektor email pada BlueTape ?
\item Bagaimana cara menangkap email pengumuman ?
\item Bagaimana cara membedakan email pengumuman dengan email lain ?
\item Bagaimana rancangan basis data kolektor pengumuman ?
\item Bagaimana cara melakukan push notification pada akun Line@ ?
\item Bagaimana cara mengubah BlueTape agar dapat mendukung fitur kolektor pengumuman dengan bantuan Heroku dan PostgreSQL ?
\end{itemize}

\section{Tujuan}
\label{sec:tujuan}
\begin{itemize}
\item Membuat fitur kolektor pengumuman untuk BlueTape
\item Memilah email pengumuman dari email lain yang masuk ke email mahasiswa
\item Mengumpulkan email-email pengumuman dan memasukkannya pada basis data
\item Merancang basis data untuk menampung email-email pengumuman
\item Melakukan push notification pada akun Line@ yang berisi subjek email dan link menuju pengumuman
\item Melakukan perawatan pada BlueTape agar dapat mendukung fitur kolektor pengumuman dengan bantuan Heroku dan PostgreSQL
\end{itemize}

\section{Batasan Masalah}
\label{sec:batasan}
Pada skripsi ini masalah dibatasi dengan batasan-batasan sebagai berikut :
\begin{itemize}
\item Sistem ini hanya akan mengumpulkan email-email pengumuman yang masuk ke email studentportal (email yang disediakan oleh Unpar untuk mahasiswa)
\item Sistem ini dibuat hanya untuk jurusan teknik informatika Unpar, mahasiswa dari jurusan lain tidak dapat menggunakan sistem ini
\end{itemize}

\dtext{8}

\section{Metodologi}
\label{sec:metodepenelitian}
Metode penelitian pada skripsi ini sebagai berikut :
	\begin{enumerate}
		\item Melakukan studi literatur tentang GMail, Line, Heroku dan PostgreSQL
		\item Memodifikasi BlueTape sehingga dapat menangkap email yang masuk ke email mahasiswa
		\item Memodifikasi BlueTape sehingga dapat melakukan push notification ke akun Line@
		\item Memodifikasi BlueTape sehingga dapat berjalan di Heroku menggunakan PostgreSQL
		\item Melakukan pengujian
		\item Menulis dokumen skripsi
	\end{enumerate}

\dtext{9}

\section{Sistematika Pembahasan}
\label{sec:sispem}
\begin{enumerate}
\item Bab 1 : Pendahuluan
Bab ini membahas gambaran umum dari skripsi.
\item Bab 2 : Dasar Teori
Bab ini membahas dasar teori yang mendukung pembuatan skripsi ini.
\item Bab 3 : Analisis
Bab ini membahas analisis yang dilakukan terhadap masalah yang diusung dalam skripsi ini.
\item Bab 4 : Perancangan
Bab ini membahas perancangan sistem yang dibangun pada skripsi ini.
\item Bab 5 : Implementasi dan Pengujian
Bab ini membahas hasil implementasi yang dilakukan beserta pengujian sistem.
\item Bab 6 : Kesimpulan dan saran
Bab ini berisi kesimpulan yang didapat dari penelitian dan saran oleh penulis kepada pembaca yang hendak melanjutkan penelitian ini. 
\end{enumerate}

\dtext{10}
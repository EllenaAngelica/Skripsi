\lstset{
  basicstyle=\ttfamily,
  columns=fullflexible,
  frame=single,
  breaklines=true,
  showlines=true,
  postbreak=\mbox{\textcolor{red}{$\hookrightarrow$}\space},
}

\chapter{Implementasi dan Pengujian}
\label{chap:implementasiDanPengujian}

Bab ini membahas proses implementasi dan proses pengujian fitur Kolektor Pengumuman Informatika.
\section{Implementasi}
Bagian ini membahas implementasi dari perancangan yang telah dilakukan di Bab 3.
\subsection{Lingkungan Pengembangan}
Berikut spesifikasi perangkat keras dan perangkat lunak yang dipakai untuk pengembangan pada skripsi ini :

\textbf{Spesifikasi Perangkat keras}
\begin{itemize}
\item Processor Intel® Celeron(R) CPU 1007U @ 1.50GHz x 2 
\item Graphics Intel® Ivybridge Mobile
\item RAM 8 GB
\item Harddisk 500GB SATA
\item Wireless keyboard and mouse combo Logitech MK215
\end{itemize}

\textbf{Spesifikasi Perangkat lunak}
\begin{itemize}
\item Sistem Operasi Ubuntu 18.04.1 LTS 64-bit
\item Visual Studio Code version 1.31.0
\item apache2 -v
\item PHP 7.2.10 (cli)
\item Composer version 1.7.2
\item pgAdmin4 version 2.1 (Application Mode : Desktop)
\item psql (PostgreSQL) 10.6
\item heroku/7.21.0 linux-x64 node-v11.9.0
\end{itemize}

\subsection{Implementasi Basis Data}
Pada pembangunan fitur Kolektor Pengumuman Informatika, ada dua tabel yang ditambahkan. Kedua tabel itu adalah tabel Pengumuman dan tabel PengumumanLineFollowers. Pembuatan tabel menggunakan dua file migration terpisah : file 20181011103200\_Pengumuman\_Initial.php dan 20190210224400\_PengumumanLineFollowers\_initial.php.

\subsection{Implementasi Kelas}
Pada pembangunan fitur Kolektor Pengumuman Informatika, dibuat kelas-kelas berikut : 
\begin{itemize}
\item Kelas model Pengumuman\_model

Pengumuman\_model merupakan kelas yang berisi algoritma yang dibutuhkan oleh fitur Kolektor Pengumuman Informatika.

\item Kelas model Pengumuman\_Line\_model

Pengumuman\_model merupakan kelas yang dikhususkan untuk algoritma untuk menghubungkan BlueTape dengan LINE API.

\item Kelas controller Cron

Cron merupakan kelas yang berfungsi untuk menjalankan perintah-perintah yang harus dijalankan pada jadwal tertentu. Pada skripsi ini perintah yang dijadwalkan adalah memeriksa email. Pada tahap perancangan, perintah untuk memeriksa email dijadwalkan tiap lima belas menit. Namun, karena keterbatasan dana, perintah ini dijadwalkan tiap hari pada jam 12 tepat siang.

\item Kelas controller Pengumuman

Pengumuman merupakan kelas yang berfungsi untuk mengatur hubungan antara Pengumuman\_model dan view yang ada di package Pengumuman.

\item Kelas controller PengumumanLine

Pengumuman\_Line merupakan kelas yang berfungsi untuk menerima webhook dari LineAPI.

\end{itemize}

Untuk mendukung kinerja kelas-kelas tersebut, dibuat juga file :
\begin{itemize}
\item File view main.php.

File ini digunakan untuk mengatur tampilan halaman utama Pengumuman.

\item File view read.php.

File ini digunakan untuk mengatur tampilan halaman saat detail pengumuman ditampilkan.

\item File config pengumuman.php.

File ini digunakan untuk menyimpan daftar pengirim pengumuman yang terverifikasi.

\item File migration 20181011103200\_Pengumuman\_Initial.php 

File ini digunakan untuk membuat tabel Pengumuman.

\item File migration 20190210224400\_PengumumanLineFollowers\_initial.php.

File ini digunakan untuk membuat tabel PengumumanLineFollowers.
\end{itemize}

Selain itu, ada beberapa file yang harus diubah :
\begin{itemize}
\item file config database.php

Pada file ini informasi database disesuaikan dengan informasi database yang dipakai di skripsi ini.

\item file config modules.php

Pada file ini ditambahkan modules Pengumuman pada config 'modules'.

\item file config routes.php

Pada file ini ditambahkan routing berikut :
\begin{lstlisting}
$route['pengumuman/page-(:num)'] = '/pengumuman/page/$1';
\end{lstlisting}
\end{itemize}

\section{Pengujian}
\subsection{Lingkungan Pengujian}
Berikut spesifikasi yang dipakai untuk pengujian pada skripsi ini :
\begin{itemize}
\item Heroku dengan spesifikasi :

\begin{itemize}
\item Region : United States
\item Stack : heroku-18
\item Framework : PHP
\item Maximum Slug Size : 500 MiB
\item Heroku Git URL : https://git.heroku.com/shadowtape.git
\item Buildpack : heroku/php
\item Domain : https://shadowtape.herokuapp.com/
\item Dyno Type : Free Dynos
\item Add-ons Heroku Postgres dan Heroku Scheduler
\end{itemize}

\item Bot LINE dengan plan Developer
\end{itemize}

\subsection{Pengujian Fungsional}
Pengujian fungsional dilakukan dengan metode black box testing. Berikut adalah hasil pengujiannya :
\begin{itemize}
  \item Pengujian Filter Email Pengumuman

    Pengujian ini bertujuan untuk menguji apakah filter email pengumuman berfungsi dengan baik. Email yang dikirim di pengujian ini memiliki subjek. Hasil pengujian dapat dilihat di Tabel \ref{table:pengujian-fungsional-filter-email}.

    \begin{center}
      \begin{table}[H]
        \caption{Pengujian Filter Email Pengumuman}
        \label{table:pengujian-fungsional-filter-email}
        \begin{tabular}{|p{5cm}|p{5cm}|p{5cm}|}
        \hline
        \centering Aksi	& 	\centering Reaksi yang diharapkan &  \multicolumn{1}{c|}{Reaksi Perangkat Lunak} \\
        \hline
        Mengirimkan email dengan email yang terdaftar lalu menjalankan Cron. & Email masuk ke database dan pengumuman ditampilkan di menu pengumuman. & Reaksi sesuai dengan yang diharapkan. Email masuk ke database dan pengumuman dapat dilihat di menu pengumuman. \\
        \hline
        Mengirimkan email dengan email yang tidak terdaftar lalu menjalankan Cron. & Email tidak masuk ke database. & Reaksi sesuai dengan yang diharapkan. Email tidak masuk ke database. \\
        \hline
        \end{tabular}
    \end{table}
    \end{center}

  \item Pengujian Mengirim Email dengan Isi Email yang Variatif

    Pengujian ini bertujuan untuk menguji apakah isi email yang ditampilkan sesuai dengan yang diharapkan. Pengujian ini dilakukan dengan mengirimkan beberapa email dengan isi yang berbeda melalui email yang terdaftar di BlueTape. Hasil pengujian dapat dilihat di Tabel ~\ref{table:pengujian-fungsional-isi-variatif}.

      \begin{longtable}{|p{5cm}|p{5cm}|p{5cm}|}
        \caption{Pengujian Mengirim Email dengan Isi Email yang Variatif}
        \label{table:pengujian-fungsional-isi-variatif}\\
        \hline
        \centering Aksi	& 	\centering Reaksi yang diharapkan &  \multicolumn{1}{c|}{Reaksi Perangkat Lunak} \\
        \hline
        Mengirimkan email tanpa subjek lalu menjalankan Cron. & Email tidak masuk ke database. & Reaksi sesuai dengan yang diharapkan. Email tidak masuk ke database. \\
        \hline
        Mengirimkan email tanpa isi lalu menjalankan Cron. & Email masuk ke database dan pengumuman ditampilkan di menu pengumuman. & Reaksi sesuai dengan yang diharapkan. Email masuk ke database dan pengumuman dapat dilihat di menu pengumuman. \\
        \hline
        Mengirimkan email dengan subjek dan isi lalu menjalankan Cron. & Email masuk ke database dan pengumuman ditampilkan di menu pengumuman. & Reaksi sesuai dengan yang diharapkan. Email masuk ke database dan pengumuman dapat dilihat di menu pengumuman. \\
        \hline
        Mengirimkan email balasan lalu menjalankan Cron. & Email masuk ke database dan pengumuman ditampilkan di menu pengumuman. & Reaksi sesuai dengan yang diharapkan. Email masuk ke database dan pengumuman dapat dilihat di menu pengumuman. \\
        \hline
        Mengirimkan email terusan lalu menjalankan Cron. & Email masuk ke database dan pengumuman ditampilkan di menu pengumuman. & Reaksi sesuai dengan yang diharapkan. Email masuk ke database dan pengumuman dapat dilihat di menu pengumuman. \\
        \hline
        Mengirimkan email dengan lampiran lalu menjalankan Cron. & Email masuk ke database dan pengumuman ditampilkan di menu pengumuman. Di bawah isi pengumuman, ada keterangan "*) Pengumuman ini memiliki lampiran, silahkan memeriksa langsung email student Anda untuk mengunduhnya.". & Reaksi sesuai dengan yang diharapkan. Email masuk ke database dan pengumuman dapat dilihat di menu pengumuman. Di bawah isi pengumuman, keterangan "*) Pengumuman ini memiliki lampiran, silahkan memeriksa langsung email student Anda untuk mengunduhnya." berhasil ditampilkan. \\
        \hline
        Mengirimkan email yang terdapat sisipan lampiran berupa gambar di isi email lalu menjalankan Cron. & Email masuk ke database dan pengumuman ditampilkan di menu pengumuman. Di bawah isi pengumuman, ada keterangan "*) Pengumuman ini memiliki lampiran, silahkan memeriksa langsung email student Anda untuk mengunduhnya.". Isi pesan harus masih lengkap walaupun gambar tidak akan berhasil ditampilkan (karena file gambar tidak bisa disimpan di server). & Reaksi sesuai dengan yang diharapkan. Email masuk ke database dan pengumuman dapat dilihat di menu pengumuman. Di bawah isi pengumuman, keterangan "*) Pengumuman ini memiliki lampiran, silahkan memeriksa langsung email student Anda untuk mengunduhnya." berhasil ditampilkan. Isi pesan lengkap. Sesuai ekspektasi, gambar tidak bisa ditampilkan. Namun, ada keterangan alt yang berisi nama file.\\
        \hline
        Mengirimkan email yang isinya memakai berbagai jenis pemformatan yang bisa dilakukan di gmail, emoji yang disediakan gmail, dan sisipan url. Setelah itu menjalankan Cron. & Email masuk ke database dan pengumuman ditampilkan di menu pengumuman. Isi email lengkap. Pemformatan tetap sama. Emoji tidak diharapkan bisa ditampilkan. Sisipan url dapat ditampilkan dan url dapat dikunjungi. & Reaksi sesuai dengan yang diharapkan. Email masuk ke database dan pengumuman dapat dilihat di menu pengumuman. Isi email lengkap. Pemformatan tetap sama. Emoji dapat ditampilkan. Beberapa emoji berubah bentuk namun tetap memiliki bentuk yang sama. Beberapa emoji persis sama dengan yang ada di isi email yang asli. Sisipan url dapat ditampilkan dan url dapat dikunjungi. \\
        \hline
    \end{longtable}

    \item Pengujian Notifikasi LINE

    Pengujian ini bertujuan untuk menguji apakah notifikasi LINE dapat terkirim ke bot LINE. Tabel \ref{table:pengujian-fungsional-notifikasi-line}.

    \begin{center}
      \begin{table}[H]
        \caption{Pengujian Notifikasi LINE}
        \label{table:pengujian-fungsional-notifikasi-line}
        \begin{tabular}{|p{5cm}|p{5cm}|p{5cm}|}
        \hline
        \centering Aksi	& 	\centering Reaksi yang diharapkan &  \multicolumn{1}{c|}{Reaksi Perangkat Lunak} \\
        \hline
        Follow akun bot. & User Id LINE tercatat di database. & Reaksi sesuai dengan yang diharapkan. User Id LINE berhasil tercatat di database.\\
        \hline 
        Mengirimkan email dengan email yang terdaftar lalu menjalankan Cron. Email harus memiliki subjek. & Setelah Cron sukses dijalankan, notifikasi LINE dari akun bot muncul. & Reaksi sesuai dengan yang diharapkan. Notifikasi LINE dari akun bot muncul. \\
        \hline
        Membuka pesan LINE yang masuk. Setelah itu membuka url yang tercantum di pesan. & URL dapat dibuka. Apabila belum login ke BlueTape, pengguna akan diarahkan ke menu login. Setelah login, pengguna diarahkan kembali ke URL tersebut. & Reaksi sesuai dengan yang diharapkan. URL dapat dibuka. Apabila belum login ke BlueTape, pengguna akan diarahkan ke menu login. Setelah login, pengguna diarahkan kembali ke URL tersebut.  \\
        \hline
        \end{tabular}
    \end{table}
    \end{center}
\end{itemize}


\subsection{Pengujian Eksperimental}
Pengujian eksperimental dilakukan dengan melibatkan partisipasi dari kelas ADPL sebagai penguji. Berikut langkah - langkah pengujian eksperimental fitur Kolektor Pengumuman Informatika :
\begin{enumerate}
\item Mahasiswa kelas ADPL mengikuti bot BlueTape dengan menambahkan id LINE "@ibz3613t" sebagai teman atau menambahkan melalui kode QR (Gambar~\ref{fig:kode-qr-bot-shadowtape}).

\begin{figure}[H]
	\centering  
	\includegraphics[scale=0.25]{kode-qr-bot-shadowtape.png}  
	\caption[Kode QR Bot Shadowtape]{Kode QR Bot Shadowtape} 
	\label{fig:kode-qr-bot-shadowtape} 
\end{figure}

\item Dosen kelas ADPL mengirimkan email berisi pengumuman menggunakan alamat email yang terdaftar di BlueTape, yaitu alamat email dengan domain unpar.
\item Mahasiswa kelas ADPL menerima notifikasi LINE tentang pengumuman tersebut jam 12 siang setelah email dikirim.
\item Mahasiswa kelas ADPL membuka url yang dikirimkan melalui notifikasi LINE.
\item Mahasiswa kelas ADPL perlu login terlebih dahulu menggunakan email student miliknya sebelum diarahkan kembali ke url pengumuman.
\item Pengujian dilakukan selama 1 minggu.
\item Setelah masa pengujian habis, dosen dan mahasiswa mengisi kuesioner.
\end{enumerate}
\chapter{Implementasi dan Pengujian}
\label{chap:implementasiDanPengujian}

\section{Implementasi}
\subsubsection{Lingkungan Pengembangan dan Pengujian Fungsional}
Berikut spesifikasi perangkat keras dan perangkat lunak yang dipakai pada skripsi ini :

\textbf{Spesifikasi Perangkat keras}
\begin{itemize}
\item Processor Intel® Celeron(R) CPU 1007U @ 1.50GHz x 2 
\item Graphics Intel® Ivybridge Mobile
\item RAM 8 GB
\item Harddisk 500GB SATA
\item Wireless keyboard and mouse combo Logitech MK215
\end{itemize}

\textbf{Spesifikasi Perangkat lunak}
\begin{itemize}
\item Sistem Operasi Ubuntu 18.04.1 LTS 64-bit
\item Visual Studio Code version 1.31.0
\item apache2 -v
\item PHP 7.2.10 (cli)
\item Composer version 1.7.2
\item pgAdmin4 version 2.1 (Application Mode : Desktop)
\item psql (PostgreSQL) 10.6
\item heroku/7.21.0 linux-x64 node-v11.9.0
\end{itemize}

\subsubsection{Implementasi Basis Data}
Pada pembangunan fitur kolektor pengumuman, ada dua tabel yang ditambahkan. Kedua tabel itu adalah tabel Pengumuman dan tabel Line\_followers. Pembuatan tabel menggunakan dua file migration terpisah.

\subsection{Implementasi Kelas}

\section{Pengujian}
\lstset{
  basicstyle=\ttfamily,
  columns=fullflexible,
  frame=single,
  breaklines=true,
  showlines=true,
  postbreak=\mbox{\textcolor{red}{$\hookrightarrow$}\space},
}

\chapter{Implementasi dan Pengujian}
\label{chap:implementasiDanPengujian}

Bab ini membahas proses implementasi dan proses pengujian fitur Kolektor Pengumuman Informatika.
\section{Implementasi}
Bagian ini membahas implementasi dari perancangan yang telah dilakukan di Bab 3.
\subsection{Lingkungan Pengembangan dan Pengujian Fungsional}
Berikut spesifikasi perangkat keras dan perangkat lunak yang dipakai pada skripsi ini :

\textbf{Spesifikasi Perangkat keras}
\begin{itemize}
\item Processor Intel® Celeron(R) CPU 1007U @ 1.50GHz x 2 
\item Graphics Intel® Ivybridge Mobile
\item RAM 8 GB
\item Harddisk 500GB SATA
\item Wireless keyboard and mouse combo Logitech MK215
\end{itemize}

\textbf{Spesifikasi Perangkat lunak}
\begin{itemize}
\item Sistem Operasi Ubuntu 18.04.1 LTS 64-bit
\item Visual Studio Code version 1.31.0
\item apache2 -v
\item PHP 7.2.10 (cli)
\item Composer version 1.7.2
\item pgAdmin4 version 2.1 (Application Mode : Desktop)
\item psql (PostgreSQL) 10.6
\item heroku/7.21.0 linux-x64 node-v11.9.0
\end{itemize}

\subsection{Implementasi Basis Data}
Pada pembangunan fitur Kolektor Pengumuman Informatika, ada dua tabel yang ditambahkan. Kedua tabel itu adalah tabel Pengumuman dan tabel PengumumanLineFollowers. Pembuatan tabel menggunakan dua file migration terpisah : file 20181011103200\_Pengumuman\_Initial.php dan 20190210224400\_PengumumanLineFollowers\_initial.php.

\subsection{Implementasi Kelas}
Pada pembangunan fitur Kolektor Pengumuman Informatika, dibuat kelas-kelas berikut : 
\begin{itemize}
\item Kelas model Pengumuman\_model

Pengumuman\_model merupakan kelas yang berisi algoritma yang dibutuhkan oleh fitur Kolektor Pengumuman Informatika.

\item Kelas model Pengumuman\_Line\_model

Pengumuman\_model merupakan kelas yang dikhususkan untuk algoritma untuk menghubungkan BlueTape dengan LINE API.

\item Kelas controller Cron

Cron merupakan kelas yang berfungsi untuk menjalankan perintah-perintah yang harus dijalankan pada jadwal tertentu. Pada skripsi ini perintah yang dijadwalkan adalah memeriksa email tiap hari.

\item Kelas controller Pengumuman

Pengumuman merupakan kelas yang berfungsi untuk mengatur hubungan antara Pengumuman\_model dan view yang ada di package Pengumuman.

\item Kelas controller PengumumanLine

Pengumuman\_Line merupakan kelas yang berfungsi untuk menerima webhook dari LineAPI.

\end{itemize}

Untuk mendukung kinerja kelas-kelas tersebut, dibuat juga file :
\begin{itemize}
\item File view main.php.

File ini digunakan untuk mengatur tampilan halaman utama Pengumuman.

\item File view read.php.

File ini digunakan untuk mengatur tampilan halaman saat detail pengumuman ditampilkan.

\item File config pengumuman.php.

File ini digunakan untuk menyimpan daftar pengirim pengumuman yang terverifikasi.

\item File migration 20181011103200\_Pengumuman\_Initial.php 

File ini digunakan untuk membuat tabel Pengumuman.

\item File migration 20190210224400\_PengumumanLineFollowers\_initial.php.

File ini digunakan untuk membuat tabel PengumumanLineFollowers.
\end{itemize}

Selain itu, ada beberapa file yang harus diubah :
\begin{itemize}
\item file config database.php

Pada file ini informasi database disesuaikan dengan informasi database yang dipakai di skripsi ini.

\item file config modules.php

Pada file ini ditambahkan modules Pengumuman pada config 'modules'.

\item file config routes.php

Pada file ini ditambahkan routing berikut :
\begin{lstlisting}
$route['pengumuman/page-(:num)'] = '/pengumuman/page/$1';
\end{lstlisting}
\end{itemize}

\section{Pengujian}
\chapter{Perancangan dan Implementasi}
\label{chap:perancanganDanImplementasi}

\section{Perancangan Kelas}


\section{Perancangan Basis Data}
Tabel-tabel yang sudah ada di BlueTape tidak diubah, namun ada dua tabel yang ditambahkan. Kedua tabel tersebut adalah tabel Pengumuman dan tabel Line\_followers. Tabel pengumuman berguna untuk menyimpan informasi dari email pengumuman. Sedangkan tabel Line\_followers berguna untuk menyimpan user id dari follower akun bot BlueTape.

\subsubsection{Tabel pengumuman}
\begin{center}
	\begin{table}[H]
	\caption{Perancangan Tabel jadwal\_dosen}
	\begin{tabular}{|c|c|c|c|c|c|}
 			\hline
			\textbf{Atribut} & \textbf{Tipe Data} & \textbf{Constraint} & \textbf{PK*}  & \textbf{FK*} \\
			\hline
		 	 id & int & - & Ya & Tidak\\
			\hline
			 namaPengirim & VARCHAR & 256 & Tidak & Tidak\\
            \hline
			 emailPengirim & VARCHAR & 256 & Tidak & Tidak\\
            \hline
			 waktuTerkirim & timestamp & - & Tidak & Tidak\\
            \hline
			 subjek & VARCHAR & 256 & Tidak & Tidak\\
            \hline
			 isi & TEXT & 256 & Tidak & Tidak\\
            \hline
			 ketersediaanLampiran & VARCHAR & 1 & Tidak & Tidak\\
			\hline
	\end{tabular}
	\end{table}
\end{center}
\textit{*PK = Primary Key} \\
\textit{*FK = Foreign Key} \\

Keterangan atribut :
\begin{itemize}
\item \textbf{id} : Id pengumuman. Auto increment.
\item \textbf{namaPengirim} : Nama pengirim email pengumuman.
\item \textbf{emailPengirim} : Alamat email pengirim email pengumuman.
\item \textbf{waktuTerkirim} : Waktu terkirim email pengumuman.
\item \textbf{subjek} : Subjek email pengumuman.
\item \textbf{isi} : Isi email pengumuman. Boleh kosong.
\item \textbf{ketersediaanLampiran} : Jika email pengumuman memiliki lampiran maka atribut ini memiliki value 'Y'. Jika tidak, maka valuenya 'N'.
\end{itemize}

\subsubsection{Tabel Line\_followers}
\begin{center}
	\begin{table}[H]
	\caption{Perancangan Tabel jadwal\_dosen}
	\begin{tabular}{|c|c|c|c|c|c|}
 			\hline
			\textbf{Atribut} & \textbf{Tipe Data} & \textbf{Constraint} & \textbf{PK*}  & \textbf{FK*} \\
			\hline
			 userId & VARCHAR & 256 & Ya & Tidak\\
            \hline
	\end{tabular}
	\end{table}
\end{center}
\textit{*PK = Primary Key} \\
\textit{*FK = Foreign Key} \\

Keterangan atribut :
\begin{itemize}
\item \textbf{userId} : User id dari akun LINE yang follow akun bot BlueTape.
\end{itemize}

\section{Perancangan Antarmuka}
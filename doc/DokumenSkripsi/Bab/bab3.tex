\lstset{
  basicstyle=\ttfamily,
  columns=fullflexible,
  frame=single,
  breaklines=true,
  showlines=true,
  postbreak=\mbox{\textcolor{red}{$\hookrightarrow$}\space},
}

\chapter{Analisis}
\label{chap:analisis}
	Pengumpulan data dalam skripsi ini dilakukan dengan cara studi pustaka.

\section{Analisis Sistem yang Sudah Ada}
\label{sec:analisiskini}
\begin{figure}[H]
	\centering  
	\includegraphics[width=\textwidth]{bluetape-login.png}  
	\caption[Tampilan utama BlueTape]{Tampilan utama BlueTape} 
	\label{fig:bluetape-login} 
\end{figure}

	\textit{BlueTape} adalah perangkat lunak yang berfungsi untuk membantu urusan-urusan \textit{paper-based} di FTIS UNPAR menjadi \textit{paperless}. Pada saat skripsi ini dibuat, BlueTape dapat diakses melalui situs web \url{https://bluetape.azurewebsites.net/} (Gambar~\ref{fig:bluetape-login}). Perangkat lunak ini bersifat open source, sehingga kode program BlueTape bisa dipelajari, diubah, dan distribusi oleh siapapun untuk tujuan apapun. Kode program ini dapat diakses di \url{https://github.com/ftisunpar/BlueTape}. BlueTape memanfaatkan CodeIgniter (versi 3.1.4) dan ZURB Foundation.
	
	Pola pengembangan yang dipakai BlueTape adalah MVC (Model-View-Controller). MVC (Model-View-Controller) adalah sebuah metode untuk membuat perangkat lunak menjadi tiga bagian : Model, View, dan Controller. Model adalah kelas yang merepresentasikan struktur data. View adalah informasi yang disajikan ke pengguna. Controller adalah penghubung antara Model, View, dan sumber daya lain yang dibutuhkan untuk mengolah HTTP request dan menghasilkan situs web.

\subsection{Aturan Konstribusi BlueTape}
	Terdapat beberapa aturan apabila ingin berkonstribusi pada pengembangan BlueTape. Aturan-aturan tersebut tertera pada dokumen \texttt{CONTRIBUTING.md} (\url{https://github.com/ftisunpar/BlueTape/blob/master/CONTRIBUTING.md}).

	\subsubsection{Pengelompokan Module}
		Perangkat lunak BlueTape dikelompokkan dalam module. Setiap module memiliki nama yang mengikuti aturan CamelCase. Jika beberapa module tergabung pada satu topik yang sama, topik tersebut harus digunakan sebagai kata pertama dalam penamaan module. Contohnya : apabila nama topik adalah \texttt{Transkrip}, maka contoh nama modulenya adalah \texttt{TranskripRequest} dan \texttt{TranskripManage}.
		
		Penamaan dokumen pada controller, view, model, config file, nama tabel, dan migration script menggunakan nama module atau topik. Contoh penamaan :
		\begin{itemize}
			\item Controller: \texttt{controllers/TranskripRequest.php}, \texttt{controllers/TranskripManage.php}
			\item View: \texttt{views/TranskripRequest/*.php}, \texttt{views/TranskripManage/*.php}
			\item Model (opsional): \texttt{models/Transkrip/*\_model.php}
			\item Config file (opsional): \texttt{config/Transkrip.php}
			\item Nama tabel (opsional): \texttt{Transkrip}
			\item Migration script (opsional): \texttt{migrations/20160222120000\_Transkrip\_initial.php}
		\end{itemize} 
	
	\subsubsection{Model}
		Model dibuat hanya jika fungsi-fungsi di dalamnya digunakan lebih dari sekali. Apabila hanya digunakan sekali, letakkan fungsi pada controller.
	
	\subsubsection{Library \texttt{bluetape}}
		Library \texttt{bluetape} berisi fungsi-fungsi yang umum digunakan di BlueTape. Contoh : fungsi untuk konversi email ke NPM.
	
	\subsubsection{Hak Akses}
		Hak akses dan nama module diatur pada dokumen \texttt{config/modules.php}. Contoh :
\begin{lstlisting}
$config['module-names'] = array(
    'TranskripRequest' => 'Permohonan Cetak Transkrip',
    'TranskripManage' => 'Manajemen Cetak Transkrip'
);

$config['modules'] = array(
    'TranskripRequest' => array('root', 'mahasiswa.ftis'),
    'TranskripManage' => array('root', 'tu.ftis')
);

$config['roles'] = array(
    'root' => 'pascal@unpar\\.ac\\.id',
    'tu.ftis' => '(shao\\.wei)@unpar\\.ac\\.id',
    'mahasiswa.ftis' => '7[123]\\d{5}@student\\.unpar\\.ac\\.id'
);
\end{lstlisting}
	
		Apabila diperlukan, kontributor boleh menambahkan role baru pada array config "roles". Setiap elemen array memetakan role dengan alamat email yang tergabung dalam role tersebut, dengan notasi regular expression.

\subsection{Autentikasi}
	Setiap module wajib memeriksa hak akses sebelum ditampilkan. Hal tersebut dilakukan dengan cara memanfaatkan template berikut pada controller:
\begin{lstlisting}
<?php
defined('BASEPATH') OR exit('No direct script access allowed');
class NamaPage extends CI_Controller {

    public function __construct() {
        parent::__construct();
        try {
            $this->Auth_model->checkModuleAllowed(get_class());
        } catch (Exception $ex) {
            $this->session->set_flashdata('error', $ex->getMessage());
            header('Location: /');
        }
    }

    // ... implementasikan method-method Anda yang lain di sini...
}
\end{lstlisting}

\subsubsection{View}
	Setiap view menggunakan template yang menampilkan nama module, menu navigasi, dan \textit{flash message} (jika diperlukan). Setiap view membutuhkan parameter \texttt{currentModule}, selain parameter-parameter lainnya. Jika ingin memanggil view dari controller, fungsi \texttt{get\_class()} dapat digunakan. Berikut adalah cara sederhana memanggil view :  
\begin{lstlisting}
$this->load->view('NamaPage/main', array('currentModule' => get_class()));
\end{lstlisting}

	View memanfaatkan framework Zurb Foundation, dan berisi template menu utama serta flash message. Oleh karena itu, kode berikut digunakan untuk memulai membuat view :
\begin{lstlisting}
<?php
defined('BASEPATH') OR exit('No direct script access allowed');
?><!doctype html>
<html class="no-js" lang="en">
    <?php $this->load->view('templates/head_loggedin'); ?>
    <body>
        <?php $this->load->view('templates/topbar_loggedin'); ?>
        <?php $this->load->view('templates/flashmessage'); ?>

        <!-- Tulislah isi view Anda di sini. -->

        <script src="/public/foundation-6/js/vendor/jquery.min.js"></script>
        <script src="/public/foundation-6/js/vendor/what-input.min.js"></script>
        <script src="/public/foundation-6/js/foundation.min.js"></script>
        <script src="/public/foundation-6/js/app.js"></script>
    </body>
</html>
\end{lstlisting}

\subsection{Fitur - Fitur BlueTape}
	Saat skripsi ini ditulis, BlueTape memiliki perangkat lunak BlueTape memiliki tiga layanan, yaitu Transkrip \textit{Request} / \textit{Manage}, Perubahan Kuliah \textit{Request} / \textit{Manage}, dan perekam jadwal dosen. Layanan Transkrip \textit{Request} / \textit{Manage} memberikan layanan untuk melakukan permohonan serta pencetakan transkrip mahasiswa. Layanan Perubahan Kuliah \textit{Request} / \textit{Manage} memberikan layanan untuk permohonan dan pencetakan perubahan jadwal kuliah oleh dosen. Layanan perekam jadwal dosen untuk merekam dan menampilkan jadwal dosen.

	\subsubsection{Transkrip \textit{Request}} 
	\begin{figure}[H]
		\centering  
		\includegraphics[width=\textwidth]{bluetape-cetak-transkrip.png}  
		\caption[Tampilan Cetak Transkrip]{Tampilan Cetak Transkrip} 
		\label{fig:bluetape-cetak-transkrip} 
	\end{figure}
	
	Gambar~ \ref{fig:bluetape-cetak-transkrip} menampilkan halaman utama saat cetak transkrip. Halaman ini hanya bisa diakses oleh user yang termasuk roles 'root' dan 'mahasiswa.ftis'. Terdapat form yang meminta input alamat email pemohon, npm, nama, tipe transkrip, dan keperluan. Input alamat email pemohon, npm, dan nama sudah terisi otomatis dan tidak bisa diubah lagi sehingga user hanya perlu mengganti tipe transkrip dan mengisi keperluan. Tipe transkrip memiliki tiga pilihan, yaitu : DPS Bahasa Indonesia (Seluruh Semester), DPS Bahasa Inggris (Seluruh Semester), dan LHS (Semester Terakhir). Selain form tersebut, terdapat tabel histori permohonan yang akan menampilkan riwayat permohonan cetak transkrip jika sudah pernah memohon. Tabel histori permohonan memiliki tujuh kolom, yaitu : ID, Status, Tanggal Permohonan, Tipe Transkrip, Tanggal Jawab/Cetak, Keterangan, dan Aksi (tindakan yang bisa dilakukan dengan record). Gambar~ \ref{fig:bluetape-cetak-transkrip-request} menampilkan tampilan setelah form permohonan transkrip baru dikirimkan. Pada Gambar~ \ref{fig:bluetape-cetak-transkrip-request} aksi yang tersedia hanya melihat detail permohonan.

	\begin{figure}[H]
		\centering  
		\includegraphics[width=\textwidth]{bluetape-cetak-transkrip-request.png}  
		\caption[Tampilan hasil Request Cetak Transkrip]{Tampilan hasil Request Cetak Transkrip} 
		\label{fig:bluetape-cetak-transkrip-request} 
	\end{figure}
	
	\subsubsection{Transkrip \textit{Manage}}
	\begin{figure}[H]
		\centering  
		\includegraphics[width=\textwidth]{bluetape-manajemen-transkrip.png}  
		\caption[Tampilan Manajemen Transkrip BlueTape]{Tampilan Manajemen Transkrip BlueTape} 
		\label{fig:bluetape-manajemen-transkrip} 
	\end{figure}
	
	Gambar~ \ref{fig:bluetape-manajemen-transkrip} menampilkan tampilan halaman Manajemen Cetak Transkrip. Halaman ini hanya bisa diakses oleh user yang termasuk roles 'root' dan 'tu.ftis'. Halaman ini memiliki kolom pencarian yang dapat diisi dengan npm mahasiswa. Angka "2013730013" pada Gambar~ \ref{fig:bluetape-manajemen-transkrip} merupakan placeholder saja, bukan input user. Apabila input kosong, maka semua permohonan akan ditampilkan pada tabel di bawah kolom pencarian. Pada Gambar~ \ref{fig:bluetape-manajemen-transkrip}, permohonan yang masuk baru satu saja. Apabila input diisi dan tombol "Cari ditekan", maka permohonan yang ditampilkan hanya permohonan milik npm yang diinput. User dapat melakukan empat aksi untuk tiap record yang ditampilkan : melihat detail permohonan (simbol mata), menolak permohonan (simbol jempol ke bawah), menyetujui permohonan (simbol print), dan menghapus permohonan (simbol tempat sampah).
	
	\subsubsection{Perubahan Kuliah \textit{Request}}
	\begin{figure}[H]
		\centering  
		\includegraphics[width=\textwidth]{bluetape-perubahan-kuliah-request.png}  
		\caption[Tampilan request perubahan kuliah]{Tampilan request perubahan kuliah} 
		\label{fig:bluetape-perubahan-kuliah-request} 
	\end{figure}
	
	Gambar~ \ref{fig:bluetape-perubahan-kuliah-request} menampilkan halaman Perubahan Kuliah. Halaman ini hanya bisa diakses oleh user yang termasuk roles 'root' dan 'staf.unpar'. Terdapat form yang meminta input alamat email pemohon, nama, kode mk, nama mata kuliah, kelas, jenis perubahan, hari dan jam sebelum dan sesudah diubah, serta ruang sebelum dan sesudah diubah. Input alamat email pemohon dan nama sudah terisi otomatis dan tidak bisa diubah lagi. Jenis perubahan memiliki tiga pilihan, yaitu : diganti, tambahan, dan ditadakan. Selain form tersebut, terdapat tabel histori permohonan yang akan menampilkan riwayat permohonan perubahan jadwal kuliah. Tabel histori permohonan memiliki delapan kolom, yaitu : ID, Status, Tanggal Permohonan, Kode MK, Perubahan, Tanggal Jawab, Keterangan, dan Aksi (tindakan yang bisa dilakukan dengan record).
	
	\subsubsection{Perubahan Kuliah \textit{Manage}}
	\begin{figure}[H]
		\centering  
		\includegraphics[width=\textwidth]{bluetape-perubahan-kuliah-manajemen.png}  
		\caption[Tampilan manage perubahan kuliah]{Tampilan manage perubahan kuliah} 
		\label{fig:bluetape-perubahan-kuliah-manajemen} 
	\end{figure}
	
	Gambar~ \ref{fig:bluetape-perubahan-kuliah-manajemen} menampilkan halaman Manajemen Perubahan Kuliah. Halaman ini hanya bisa diakses oleh user yang termasuk roles 'root' dan 'tu.ftis'. Halaman ini berisi riwayat permohonan perubahan kuliah. User dapat melakukan empat aksi untuk tiap record yang ditampilkan : melihat detail permohonan (simbol mata), menolak permohonan (simbol jempol ke bawah), menyetujui permohonan (simbol print), dan menghapus permohonan (simbol tempat sampah).
	
	\subsubsection{Entri Jadwal Dosen}
	\begin{figure}[H]
		\centering  
		\includegraphics[width=\textwidth]{bluetape-entri-jadwal-dosen-tambah.png}  
		\caption[Tampilan tambah jadwal dosen]{Tampilan tambah jadwal dosen} 
		\label{fig:bluetape-entri-jadwal-dosen-tambah} 
	\end{figure}
	
	Gambar~ \ref{fig:bluetape-entri-jadwal-dosen-tambah}
	\begin{figure}[H]
		\centering  
		\includegraphics[width=\textwidth]{bluetape-entri-jadwal-dosen-daftar.png}  
		\caption[Tampilan jadwal dosen]{Tampilan jadwal dosen} 
		\label{fig:bluetape-entri-jadwal-dosen-daftar} 
	\end{figure}
	
	Halaman Entri Jadwal Dosen hanya bisa diakses oleh user yang termasuk roles 'root' dan 'dosen.informatika'. Halaman ini terdiri dari dua bagian : form tambah jadwal dan daftar jadwal. Gambar~ \ref{fig:bluetape-entri-jadwal-dosen-tambah} menunjukkan form tambah jadwal. Form ini meminta input hari, durasi, label, jam mulai dan jenis. Gambar~ \ref{fig:bluetape-entri-jadwal-dosen-daftar} menunjukkan daftar jadwal milik user. Kolom yang memiliki isi bisa diklik untuk diedit. Gambar~ \ref{fig:bluetape-edit-jadwal-dosen} menunjukkan tampilan edit jadwal.
	
	\begin{figure}[H]
		\centering  
		\includegraphics[width=\textwidth]{bluetape-edit-jadwal-dosen.png}  
		\caption[Tampilan edit jadwal dosen]{Tampilan edit jadwal dosen} 
		\label{fig:bluetape-edit-jadwal-dosen} 
	\end{figure}

	\subsubsection{Lihat Jadwal Dosen}
	\begin{figure}[H]
		\centering  
		\includegraphics[width=\textwidth]{bluetape-lihat-jadwal-dosen.png}  
		\caption[Tampilan lihat jadwal dosen]{Tampilan lihat jadwal dosen} 
		\label{fig:bluetape-lihat-jadwal-dosen} 
	\end{figure}
	
	Gambar~ \ref{fig:bluetape-lihat-jadwal-dosen} menampilkan halaman Lihat Jadwal Dosen. Halaman ini bisa diakses oleh user yang termasuk roles 'root', 'mahasiswa.informatika', dan 'dosen.informatika'. Halaman ini memiliki tab-tab yang masing-masing diberi label nama dosen. Di bawah tab terdapat tabel jadwal dari dosen yang tabnya aktif.

\subsection{Hak Akses}
	Hak akses dan nama module diatur pada dokumen \texttt{config/modules.php} yang terletak di dalam direktori \texttt{config}. Hak akses dikelompokkan di dalam kelompok yang disebut \textit{role}. Saat skripsi ini dibuat, hak akses dibagi ke dalam lima \textit{role} : root, mahasiswa.ftis, tu.ftis, staf.unpar, dosen.informatika, dan mahasiswa.informatika. \textit{Role} root berisi daftar alamat email dari pengembang bluetape. \textit{Role} tu.ftis berisi daftar alamat email dari tata usaha ftis. \textit{Role} mahasiswa.ftis berisi daftar alamat email dari mahasiswa ftis. \textit{Role} staf.unpar berisi daftar alamat email dari staf unpar. \textit{Role} dosen.informatika berisi daftar alamat email dari dosen informatika.

	Setiap \textit{role} memiliki batasan dalam mengakses module di BlueTape. \textit{Role} root tidak memiliki batasan dan dapat mengakses setiap module yang ada. \textit{Role} tu.ftis hanya dapat mengakses module \texttt{TranskripManage}, dan module \texttt{PerubahanKuliahManage}. \textit{Role} mahasiswa.ftis hanya dapat mengakses module \texttt{TranskripRequest} dan module \texttt{LihatJadwalDosen}. \textit{Role} staf.unpar hanya dapat mengakses module \texttt{PerubahanKuliahRequest}. \texttt{Role} dosen.informatika hanya dapat mengakses module \texttt{EntriJadwalDosen}.

\section{Analisis Fitur yang Dibangun}
\label{sec:analisisYangDibangun}
	Bagian ini membahas analisis fitur Kolektor Pengumuman Informatika.

\subsection{Diagram Use Case}
\begin{figure}[H]
	\centering  
	\includegraphics[width=\textwidth]{use-case-diagram.jpg}  
	\caption[Use case diagram fitur Kolektor Pengumuman Informatika]{Use case diagram fitur Kolektor Pengumuman Informatika} 
	\label{fig:use-case-diagram} 
\end{figure}

Gambar~\ref{fig:use-case-diagram} merupakan gambar diagram use case fitur Kolektor Pengumuman Informatika. Pada diagram use case fitur Kolektor Pengumuman Informatika terdapat tiga aktor : dosen informatika, mahasiswa informatika, dan root (admin). Berikut ini adalah penjelasan dari skenario pada diagram use case tersebut :
\begin{enumerate}
\item Menerbitkan pengumuman via email

\begin{itemize}
	\item Aktor : Dosen Informatika.
	\item Skenario Normal

	\begin{enumerate}
		\item Dosen mengirimkan email ke alamat email yang dikhususkan untuk menampung pengumuman di jurusan Teknik Informatika.
		\item BlueTape mengecek email tersebut pada periode tertentu.
		\item Jika alamat email yang dosen pakai terdaftar di BlueTape, maka BlueTape akan menampilkannya dan mengirim notifikasi ke LINE.
	\end{enumerate}
	
	\item Skenario Exception
	\begin{enumerate}
		\item Dosen mengirimkan email ke alamat email yang dikhususkan untuk menampung pengumuman di jurusan Teknik Informatika.
		\item BlueTape mengecek email tersebut pada periode tertentu.
		\item Jika alamat email yang dosen pakai tidak terdaftar di BlueTape, maka BlueTape akan mengabaikan email tersebut.
	\end{enumerate}
\end{itemize}

\item Mendaftar sebagai penerima notifikasi pengumuman di LINE

\begin{itemize}
	\item Aktor : Mahasiswa Informatika.
	\item Skenario Normal

	\begin{enumerate}
		\item Mahasiswa mengikuti bot BlueTape dengan menambahkannya sebagai teman.
		\item Bot BlueTape akan mengirim notifikasi kepada mahasiswa jika ada pengumuman baru di BlueTape.
	\end{enumerate}
\end{itemize}

\item Menerima notifikasi pengumuman di LINE

\begin{itemize}
	\item Aktor : Mahasiswa Informatika.
	\item Skenario Normal

	\begin{enumerate}
		\item Mahasiswa menerima notifikasi dari bot BlueTape saat ada pengumuman baru di BlueTape.
		\item Mahasiswa mengunjungi url yang dicantumkan di pesan dari notifikasi tersebut. 
		\item Mahasiswa perlu login menggunakan email student miliknya terlebih dahulu sebelum dapat mengunjungi url tersebut.
	\end{enumerate}
\end{itemize}

\item Melihat pengumuman terkirim di dashboard

\begin{itemize}
	\item Aktor : Dosen Informatika, Mahasiswa Informatika, dan root (admin).
	\item Skenario Normal

	\begin{enumerate}
		\item Dosen Informatika, Mahasiswa Informatika, atau root (admin) mengunjungi menu Pengumuman. Menu Pengumuman berisi daftar pengumuman yang masuk ke BlueTape. Daftar tersebut disortir dari yang paling baru.
		\item Saat salah satu pengumuman diklik, maka isi dan detail pengumuman tersebut akan ditampilkan.
	\end{enumerate}
\end{itemize}
\end{enumerate}

\subsection{Modifikasi BlueTape agar Dapat Berjalan di Heroku}
	\subsubsection{Dependensi}
		BlueTape membutuhkan dependensi tambahan agar perangkat lunak dapat dijalankan. Dependensi tambahan perangkat lunak BlueTape tertera pada dokumen \texttt{composer.json}. Berikut isinya :
		\begin{lstlisting}
		{
		    "require": {
		        "google/apiclient": "^1.0",
				"ext-imap": "*",
		        "phpoffice/phpexcel": "^1.8",
		        "linecorp/line-bot-sdk": "^3.6"
		    }
		}
		\end{lstlisting}
		
		Pada saat skripsi ini ditulis, BlueTape telah memakai dua package : package \texttt{google/apiclient} dan package \texttt{phpoffice/phpexcel}. Package \texttt{google/apiclient} adalah package yang diperlukan untuk autentikasi akun saat masuk ke BlueTape. Sedangkan package \texttt{phpoffice/phpexcel} adalah package yang digunakan untuk menghasilkan dokumen excel. Package yang ditambahkan untuk skripsi ini adalah package \texttt{ext-imap} dan \texttt{line-bot-sdk}. Package \texttt{ext-imap} digunakan untuk mengakses email. Package \texttt{line-bot-sdk} digunakan untuk menghubungkan perangkat lunak dengan layanan yang disediakan oleh LINE.
		
	\subsubsection{Tipe Proses}
		BlueTape memiliki satu tipe proses, yaitu tipe proses \texttt{web} dan tipe proses \texttt{release}. Tipe proses \texttt{web} adalah tipe proses yang digunakan untuk menerima arus HTTP eksternal dari router Heroku. Penulis tidak dapat menambahkan tipe proses lain karena itu berarti penulis harus menambah dyno. Penambahan dyno perlu informasi kartu kredit untuk verifikasi akun.
		
	\subsubsection{Procfile}
		Procfile adalah dokumen yang menjelaskan Heroku bagian-bagian perangkat lunak yang dapat dieksekusi. Procfile berisi daftar tipe proses beserta cara menjalankannya. BlueTape hanya memiliki satu tipe proses, yaitu tipe proses \texttt{web}. Isi Procfile adalah :
		\begin{lstlisting}
		web: vendor/bin/heroku-php-apache2 www/
		\end{lstlisting}
		
		Maksud dari satu baris Procfile tersebut adalah : untuk menjalankan tipe proses web, heroku harus menjalankan server apache di heroku dan kemudian server menjalankan perangkat lunak web yang ada di direktori \texttt{www}. Perintah \texttt{vendor/bin/heroku-php-apache2} adalah perintah untuk menjalankan server apache di heroku yang ada di package \texttt{heroku-php-apache2}. Package \texttt{heroku-php-apache2} ini otomatis disediakan saat membuat perangkat lunak php di heroku sehingga tidak perlu ditambahkan di composer.json. Perintah \texttt{www/} berguna untuk mengarahkan server apache heroku ke direktori \texttt{www}.
		
	\subsubsection{Slug}
		Penulis tidak menambahkan dokumen \texttt{.slugignore} karena tidak diperlukan.
		
	\subsubsection{Buildpack}
		Buildpack yang dipakai pada skripsi ini hanya \texttt{heroku/php}. Buildpack ini secara otomatis dipakai oleh Heroku karena BlueTape memakai bahasa PHP.
		
	\subsubsection{Dyno}
		Jenis dyno yang dipakai pada skripsi ini adalah free dyno. Jumlah dyno hanya satu. Dyno tersebut merupakan dyno untuk tipe proses \texttt{web}.
		
	\subsubsection{Config Vars}
			Berikut adalah config vars yang dipakai pada skripsi ini: 
		\begin{itemize}
			\item CI\_DB\_DATABASE : nama database yang digunakan.
			\item CI\_DB\_HOSTNAME : nama host dari database yang disebutkan di config var CI\_DB\_DATABASE.
			\item CI\_DB\_USERNAME : username yang digunakan untuk terhubung ke database yang disebutkan di config var CI\_DB\_DATABASE.
			\item CI\_DB\_PASSWORD : password dari username yang disebutkan di config var CI\_DB\_USERNAME.
			\item HEROKU\_POSTGRESQL\_BLUE\_URL : url database. Dibuat secara otomatis saat membuat database.
			\item GOOGLE\_CLIENTID : Google Client ID, digunakan untuk melakukan autentikasi saat user login.
			\item GOOGLE\_CLIENTSECRET : Google Client Secret, digunakan untuk melakukan autentikasi saat user login.
			\item ANNOUNCEMENT\_EMAIL : alamat email yang dipakai untuk menampung pengumuman.
			\item ANNOUNCEMENT\_PASSWORD : password untuk alamat email yang disebutkan di config var ANNOUNCEMENT\_EMAIL.
			\item HOSTNAME\_INCOMING\_EMAIL : nama host dari alamat email yang disebutkan di config var ANNOUNCEMENT\_EMAIL.
			\item CI\_BASE\_URL : base url BlueTape.
			\item LINE\_BOT\_CHANNEL\_SECRET : LINE Bot Channel Secret digunakan untuk terhubung ke channel bot untuk pengumuman.
			\item LINE\_BOT\_CHANNEL\_TOKEN : LINE Bot Channel Token digunakan untuk terhubung ke channel bot untuk pengumuman.
			\item SMTP\_HOST : SMTP host, konfigurasi untuk mengirim email.
			\item SMTP\_PASS : SMTP pass, konfigurasi untuk mengirim email.
			\item SMTP\_PORT : SMTP port, konfigurasi untuk mengirim email.
			\item SMTP\_USER : SMTP user, konfigurasi untuk mengirim email.
		\end{itemize}
		
	\subsubsection{Region}
		Region yang dipakai adalah region default, yaitu United States.
		
	\subsubsection{Stack}
		Stack yang dipakai adalah stack yang paling baru, yaitu heroku-18. Stack ini dipilih karena Heroku masa kadaluarsa dukungan untuk stack ini yang paling lama (didukung sampai bulan April 2023). Alasan lain adalah komputer lokal yang digunakan untuk mengerjakan skripsi ini menggunakan lingkungan yang mirip dengan Heroku, yaitu menggunakan Ubuntu 18.04.

	\subsubsection{Basis Data}
		Basis data yang digunakan untuk skripsi ini adalah basis data Heroku Postgres dengan plan hobby-dev. Alasan utama basis data ini digunakan adalah karena penggunaan basis data lain seperti Heroku Redis dan Apache Kafka tidak memungkinkan. Basis data tersebut membutuhkan informasi akun kredit untuk verifikasi akun. Alasan lain adalah basis data ini adalah basis data yang langsung disediakan oleh Heroku.
	
		Sebelumnya BlueTape menggunakan MySQL untuk basis datanya. Proses migrasi dari MySQL ke Heroku Postgres tidak rumit, karena menggunakan fitur Migration dari CodeIgniter. Namun, ada beberapa perubahan pada dokumen Migration. Perubahan-perubahan tersebut adalah :
	\begin{itemize}
		\item Menyelaraskan nama tabel karena sifat case sensitive pada Postgres
		\item Mengubah Replace menjadi Insert dan Update karena Replace tidak didukung oleh Postgres
		\item Mengubah tipe data kolom yang sebelumnya menggunakan DATETIME menjadi timestamp
	\end{itemize}  

\subsection{Sinkronisasi Email}
\label{sec:analisisemail}
	Awalnya sinkronisasi email dilakukan dengan memanfaatkan Gmail API. Namun, Gmail API membutuhkan token yang harus direfresh tiap periode tertentu. Sehingga penulis perlu mencari alternatif lain. Penulis memutuskan menggunakan PHP IMAP untuk melakukan sinkronisasi email.
	
	Sebelum melakukan sinkronisasi email, email khusus untuk menampung email pengumuman harus dibuat terlebih dahulu. Email dibuat melalui provider email Gmail. Setelah email selesai dibuat, fitur IMAP perlu dinyalakan terlebih dahulu.
	
	Proses sinkronisasi email dimulai dengan membuat koneksi IMAP ke email pengumuman tersebut. Dengan menggunakan koneksi IMAP yang telah didapat, email difilter dengan mencari email yang belum dibaca saja. Apabila hasil pencarian tidak kosong, maka setiap email pada hasil pencarian akan diperiksa pengirimnya. Pengirim email akan dinyatakan sebagai pemberi pengumuman yang valid apabila ia terdaftar di dalam daftar pengirim yang terverifikasi. Apabila sebuah email dinyatakan memiliki pengirim yang valid, maka informasi dari email tersebut akan diproses dan dimasukkan ke basis data.
	
	Informasi yang perlu disimpan dari email pengumuman adalah alamat email pengirim, nama pengirim, tanggal email tersebut terkirim, subjek email, isi email, dan ketersediaan lampiran. Sebuah tabel baru diperlukan untuk menampung informasi ini. Tabel ini akan diakses saat halaman pengumuman akan ditampilkan. Setiap informasi dari email tersimpan, maka satu push message akan dikirim ke LINE.
	
	Sinkronisasi email perlu dilakukan secara berkala dan otomatis. Pada skripsi ini sinkronisasi email dilakukan per hari dengan menggunakan cron dan add-on Heroku Scheduler.
	
\subsection{Menghubungkan BlueTape dengan LINE}
\label{sec:analisisline}
	Produk LINE yang digunakan untuk menghubungkan BlueTape dengan LINE adalah LINE Messaging API. Produk ini paling memenuhi kriteria fitur kolektor pengumuman, yaitu dapat mengirimkan push message.
	
	Sebelum menghubungkan BlueTape dengan LINE, ada beberapa hal yang harus dilakukan terlebih dahulu di LINE developer console. Pertama, membuat akun LINE dan mendaftar sebagai developer di LINE developer console. Kedua, membuat provider di LINE developer console. Ketiga, membuat channel pada provider tersebut. Channel yang dibuat harus menggunakan plan Developer Trial karena plan ini yang memiliki fitur push message. Setelah channel terbuat, sebuah akun bot otomatis dibuat.
	
	Pada channel, terdapat tiga bagian penting : channel access token, channel secret, dan webhook url. Channel access token dan channel secret digunakan untuk autentikasi saat aplikasi BlueTape dan LINE berinteraksi. Kedua kode ini dapat diubah-ubah menggunakan tombol issue pada masing-masing kolom. Sedangkan webhook url adalah alamat url untuk menerima POST request berisi event-event yang terjadi pada bot. Contoh : event following yang terjadi saat ada akun LINE yang follow akun bot. Webhook url dapat diubah-ubah, namun harus menggunakan alamat https. Selain itu, fitur csrf protection milik Codeigniter harus dimatikan pada alamat url tersebut.
	
	Pada skripsi ini, tidak semua event perlu ditangani. Event yang harus ditangani cukup follow event dan unfollow event. Saat follow event terjadi, user id dari akun follower akan disimpan di sebuah tabel di basis data. Apabila user id mengeblok akun bot sehingga unfollow event terjadi, maka user id tersebut dihapus dari tabel. Penyimpanan ini diperlukan karena user id diperlukan saat mengirim push message dan tidak ada fungsi pada line-bot-sdk untuk mendapatkan user id follower akun bot.
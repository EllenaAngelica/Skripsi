\chapter{Kesimpulan dan Saran}
\section{Kesimpulan}
Berikut adalah kesimpulan yang dapat ditarik dari penelitian ini :
\begin{itemize}
\item Untuk melakukan penelitian dengan biaya yang minim, peneliti harus kreatif dan berhati-hati dalam memanfaatkan sumber daya yang tersedia. Pada skripsi ini penulis harus berhati-hati memilih jenis layanan yang dipakai agar penulis tidak dikenai biaya.
\item Untuk melakukan penelitian dengan biaya yang minim, peneliti harus membatasi masalah yang membutuhkan biaya tambahan. Pada skripsi ini penulis harus membatasi masalah dengan tidak menangani lampiran. Salah satu alasannya adalah karena membutuhkan biaya tambahan untuk menyimpan dokumen lampiran.
\item Menggunakan layanan dari pihak ketiga untuk penelitian memiliki resiko, yaitu pihak penyedia layanan dapat mengubah kebijakan layanan sewaktu-waktu. Salah satu layanan pihak ketiga yang dipakai pada skripsi ini, LINE@, akan diberhentikan penggunaanya oleh LINE selaku penyedia layanan tersebut.
\item Masalah yang tak terduga dapat terjadi di penelitian. Pada skripsi ini penulis tidak menduga bahwa beberapa mahasiswa tidak dapat login karena penulis berpikir masalah login telah ditangani di repository aslinya.
\end{itemize}

\section{Saran}
Berikut adalah beberapa saran untuk pihak yang ingin melanjutkan penelitian ini :
\begin{itemize}
\item Memperbaiki tampilan agar pengguna lebih nyaman memakai fitur ini.
\item Mencoba memakai LINE Official Account.
\item Memperbaiki masalah hak akses.
\end{itemize}
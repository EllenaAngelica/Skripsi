\chapter{Kesimpulan dan Saran}
\section{Kesimpulan}
Berikut adalah kesimpulan yang dapat diambil dari penelitian ini :
\begin{itemize}
    \item Konsep dan cara kerja LINE@ dan Heroku telah dipelajari.
    \item BlueTape dapat dimodifikasi agar dapat berjalan di Heroku. BlueTape versi skripsi ini diberi nama shadowtape dan dapat diakses di \url{https://shadowtape.herokuapp.com/}.
    \item BlueTape telah memiliki fitur kolektor pengumuman. Fitur ini dapat melakukan sinkronisasi kotak masuk untuk alamat \textit{email} \href{mailto:shadowbluetape@gmail.com}{shadowbluetape@gmail.com} setiap jam. \textit{Email} yang diidentifikasi sebagai pengumuman dapat ditampilkan di \url{https://shadowtape.herokuapp.com/pengumuman}. BlueTape dapat memunculkan notifikasi pada akun LINE@ Shadowtape.
    \item Fitur kolektor pengumuman telah teruji.
\end{itemize}

\section{Saran}
Berikut adalah saran untuk pengembangan lebih lanjut :
\begin{itemize}
    \item Mengadaptasi konfigurasi LINE@ pada BlueTape ke layanan LINE Official Account.
    \item Meneliti lebih lanjut masalah \textit{login} yang ditemukan saat pengujian eksperimental.
    \item Meningkatkan usabilitas fitur ini agar calon pengguna nyaman menggunakannya.
\end{itemize}
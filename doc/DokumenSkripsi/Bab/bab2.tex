%versi 2 (8-10-2016)
\chapter{Landasan Teori}
\label{chap:teori}
Pada bab ini dijelaskan dasar-dasar teori mengenai \textit{BlueTape}, \textit{Heroku}, \textit{PostgreSQL}, \textit{GMail}, dan \textit{Line@}.

%2.1 BlueTape
\section{\textit{BlueTape}}
\label{sec:BlueTape}
\textit{BlueTape} adalah perangkat lunak yang berfungsi untuk membantu urusan-urusan \textit{paper-based} di FTIS UNPAR menjadi \textit{paperless}. Perangkat lunak ini berbasis web dengan memanfaatkan \textit{CodeIgniter} dan \textit{ZURB Foundation}. Saat ini aplikasi ini memiliki dua layanan, yaitu Transkrip \textit{Request} / \textit{Manage} dan Perubahan Kuliah \textit{Request} / \textit{Manage}. Layanan Transkrip \textit{Request} / \textit{Manage} memberikan layanan untuk melakukan permohonan serta pencetakan transkrip mahasiswa. Sedangkan layanan Perubahan Kuliah \textit{Request} / \textit{Manage} memberikan layanan untuk permohonan dan pencetakan perubahan jadwal kuliah oleh dosen. \footnotemark
\footnotetext{https://github.com/ftisunpar/BlueTape}

%2.2 Heroku
\section{\textit{Heroku}}
\label{sec:Heroku}
\textit{Heroku} adalah \textit{platform cloud} yang memungkinkan \textit{developer} untuk membangun, menjalankan, dan mengoperasikan aplikasinya pada \textit{cloud}. Bahasa yang didukung oleh \textit{Heroku} adalah \textit{Ruby}, \textit{Node.js}, \textit{Java}, \textit{Python}, \textit{Clojure}, \textit{Scala}, \textit{Go} dan \textit{PHP}. \footnotemark
\footnotetext{https://www.heroku.com/what}

\subsection{Membangun Perangkat Lunak berbasis \textit{PHP} di \textit{Heroku}}
Sebelum \textit{developer} dapat membangun perangkat lunak berbasis \textit{PHP} mereka di \textit{Heroku}, \textit{developer} harus membuat akun \textit{Heroku} terlebih dahulu. Selain itu, \textit{developer} perlu memasang \textit{PHP} dan \textit{Composer} di komputer yang digunakan untuk membangun perangkat lunak tersebut.\footnotemark
\footnotetext{https://devcenter.heroku.com/articles/getting-started-with-php}

\subsubsection{Mempersiapkan \textit{Heroku}}
Sebelum \textit{developer} dapat memulai pembangunan perangkat lunak, \textit{developer} harus sudah memasang dan mempersiapkan \textit{Git}. Setelah itu, \textit{developer} perlu memasang \textit{Heroku Command Line Interface} (CLI). \textit{Heroku} CLI digunakan untuk mengelola perangkat lunak, \textit{add-ons}, melihat \textit{log} perangkat lunak, dan menjalankan aplikasi secara lokal.\footnotemark[\value{footnote}]

Setelah \textit{Heroku} terpasang, \textit{developer} dapat menggunakan perintah \textit{Heroku} di \textit{command shell}. Perintah pertama yang harus dilakukan adalah \textit{"heroku login"}. Perintah ini berfungsi untuk otentikasi agar dapat masuk ke akun \textit{Heroku}. Otentikasi dibutuhkan agar \textit{Heroku} dan \textit{Git} dapat beroperasi. \footnotemark[\value{footnote}]

\subsubsection{Menyiapkan Perangkat Lunak}
Hal pertama yang harus dilakukan oleh \textit{developer} adalah \textit{clone} \textit{repository} ke komputer lokal. Caranya dengan mengetikkan perintah \textit{"git clone"} diikuti dengan alamat \textit{git heroku}-nya di \textit{command shell} atau \textit{terminal}. Di dalam \textit{repository} lokal sudah ada perangkat lunak sederhana dan juga \textit{file composer.json}. \textit{Heroku} menggunakan \textit{Composer} untuk mengelola \textit{dependency} dalam proyek \textit{PHP}, dan \textit{file composer.json} menjadi rujukan untuk Heroku bahwa perangkat lunak tersebut ditulis dalam bahasa \textit{PHP}. \footnotemark[\value{footnote}]

\subsubsection{\textit{Deploy} Perangkat Lunak}
Bagian ini menjelaskan cara \textit{deploy} perangkat lunak ke \textit{Heroku}. Pertama, \textit{developer} perlu membuat perangkat lunak pada \textit{Heroku}. Caranya dengan mengetikkan perintah \textit{"heroku create"} pada \textit{command shell} atau \textit{terminal}. Ketika perangkat lunak dibuat, sebuah \textit{git remote} yang otomatis dinamai \textit{heroku} juga terbentuk. \textit{Heroku} akan menamai perangkat lunak dengan nama acak yang nantinya bisa diganti. Untuk memulai deploy \textit{developer} perlu mengetikkan perintah \textit{"git push"} diikuti dengan nama \textit{git remote} lalu nama \textit{branch repository}. Contohnya : \textit{"git push heroku master"} \footnotemark[\value{footnote}]

\subsubsection{Procfile}
Procfile adalah \textit{file text} yang berfungsi untuk mendeklarasikan secara ekplisit perintah yang harus diekseskusi untuk menjalankan perangkat lunak di Heroku. Contoh isi Procfile : \textit{"web: vendor/bin/heroku-php-apache2 web/"}. Procfile dapat berisi \textit{single process type}, \textit{web}, dan perintah yang diperlukan untuk menjalankan perangkat lunak tersebut. \footnotemark[\value{footnote}]

\subsubsection{Basis Data}
Heroku menyediakan tiga layanan data : Heroku Postgres, Heroku Redis, dan Apache Kafka on Heroku. Dalam skripsi ini, layanan data yang dipakai adalah Heroku Postgres yang berbasis PostgreSQL.\footnotemark
\footnotetext{https://devcenter.heroku.com/categories/data-management}

%2.3 PostgreSQL
\section{PostgreSQL}
\label{sec:PostgreSQL}
PostgreSQL adalah sistem basis data object-relational open source. Tipe data yang didukung oleh PostgreSQL adalah :\footnotemark
\footnotetext{https://www.postgresql.org/about/}
\begin{itemize}
\item Primitives: Integer, Numeric, String, Boolean
\item Structured: Date/Time, Array, Range, UUID
\item Document: JSON/JSONB, XML, Key-value (Hstore)
\item Geometry: Point, Line, Circle, Polygon
\item Customizations: Composite, Custom Types
\end{itemize}

%2.4 GMail
\section{GMail}
\label{sec:GMail}
GMail adalah layanan surat elektronik milik Google. Salah satu fitur GMail untuk developer adalah GMail API.
\subsection{Cloud Pub/Sub}
Cloud Pub/Sub adalah fondasi awal untuk stream analytics dan event-driven computing systems. Langkah-langkah yang diperlukan untuk memperolehnya adalah\footnotemark
\footnotetext{https://cloud.google.com/pubsub/docs/quickstart-console} :
\begin{itemize}
\item Mempersiapkan GCP Console project.
\item Membuat project baru atau memilih project yang sudah ada
\item Mengaktifkan Cloud Pub/Sub API untuk project tersebut.
\item Memasang dan menginisialisasi Cloud SDK
\item Membuat Topic
\item Menambahkan subcription
\end{itemize}

\subsection{IMAP}
Agar dapat menggunakan IMAP untuk memeriksa Gmail di perangkat lunak lain, developer perlu menyiapkan dan mengubah setelan SMTP. Pertama, developer perlu masuk ke akun gmail dan pergi ke menu setelan. Di menu setelan, pilih tab Penerusan dan POP/IMAP. Pada bagian "Akses IMAP" pilih Aktifkan IMAP lalu tekan tombol "Simpan Perubahan". Pada perangkat lunak, ubah setelan SMTP dan setelan lainnya : \footnotemark
\footnotetext{https://support.google.com/mail/answer/7126229?hl=id} :
\begin{itemize}
\item Server Email Masuk (IMAP):
\begin{itemize}
\item imap.gmail.com
\item Gunakan SSL: Ya
\item Port: 993
\end{itemize}
\item Server Email Keluar (SMTP):
\begin{itemize}
\item smtp.gmail.com
\item Gunakan SSL: Ya
\item Gunakan TLS: Ya (jika tersedia)
\item Gunakan Autentikasi: Ya
\item Port untuk SSL: 465
\item Port untuk TLS/STARTTLS: 587
\end{itemize}
\item Nama Lengkap atau Nama Tampilan : Nama Anda
\item Nama Akun, Nama pengguna, atau Alamat email : Alamat email lengkap Anda
\item Sandi : Sandi Gmail Anda
\end{itemize}

%2.5 Line@.
\section{Line@}
\label{sec:Line@}
Line@ adalah layanan oleh LINE yang didesain khusus untuk bisnis atau organisasi. Line@ menyediakan berbagai fitur untuk mempromosikan suatu perusahaan, merek, atau produk dalam cara yang baru dan dengan jangkauan yang luas. Salah satu fitur tersebut adalah fitur "Message Broadcasts". Fitur ini memungkinkan pengguna mengirimkan pesan melalui perangkat lunak mobile LINE@ atau melalui perangkat lunak komputer LINE@ Manager dan menyebarkannya ke pelanggan dan fans yang telah menjadikan akun pengguna sebagai teman. LINE@ menawarkan beberapa fitur bawaan yang bisa digunakan di pesan, seperti kupon dan survei. Fitur "1-on-1 chat" memungkinkan pengguna membalas secara langsung pesan yang dikirimkan pelanggan dan fans yang menjadikan pengguna teman. Fitur "Timeline Posts" memungkinkan pengguna mengirimkan postingan di linimasa. Postingan tersebut bisa diberi "like" atau dikomentari sehingga dapat memaksimalkan potensi linimasi sebagai media pemasaran.\footnotemark
\footnotetext{https://at.line.me/en/}